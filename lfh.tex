\chapter{Logic for Humans}

You're a curious person, I suspect.  You probably already flipped
through the pages of this book, in which case you may have run across
some unfamiliar symbols.  You might have found yourself intrigued ---
like an archaeologist discovering ancient runes.  Or you might have
been put off --- thinking that this book is for quantitative people.

That's what I assumed at first.  I wanted to spend the days of my life
thinking about the big questions of human existence --- what exists,
what we can know, and how we should live.  Calculate the derivative of
a function?  Solve a differential equation?  No thank you.  I'll leave
that to the people who want to build better bridges.  I'd prefer to
move on to the really meaningful and enriching topics.

But I discovered that it's a false dilemma.  In fact, it's not a
dilemma at all.  Symbolic logic is not only for mathematics, and it's
by no means a diversion from the really deep questions of human life.
In fact, symbolic logic represents the best account we have of what it
means to be \textit{rational}.

Although logic is symbolic, it's not really ``mathematical'' in any
sense that puts it in opposition to humanistic endeavors (such as
literature, poetry, history, philosophy, etc.).  Yes, mathematics is a
human activity that displays logical thinking in a particularly clear
way.  But logic itself is involved in any type of human thinking that
aims at finding the truth.  If you've ever argued for a claim, or
evaluated someone else's argument, then you were using logic ---
whether you realized it or not, and whether or not you did a good job
of it.

The goal of this book is simple: it's to make you conscious of how you
already use logic, and thereby to become even better at it.  If you
learn to do symbolic logic, then you will become a better thinker, and
you will understand better what it means to be a good thinker.

\section{Arguments}

Many logic books begin by saying: ``The subject matter of logic is
\dots '' I think these statements are always a bit misleading.  In one
sense, logic doesn't have a subject matter at all.  Logic isn't
\textit{about} something, it's a way of life.

Let's begin by trying to see ourselves from the outside.  Just imagine
that you are an alien who has landed on earth, and you're trying to
understand what human beings are doing when they say that they are
thinking logically.  Imagine that there are two people, say Anne and
Bernt, and that Anne is trying to convince Bernt that something is
true.  Anne might proceed as follows:
\begin{quote}
  Of course gay marriage should be legal.  Only people with some
  backwards religious view would believe otherwise. \end{quote} Here
Anne is trying to convince Bernt that gay marriage should be legal.
But she doesn't try to coerce him with physical force, or even with
intellectual intimidation.  Instead, she offers Bernt a
\textit{reason} why he should accept her conclusion.  To be more
clear, the \emph{conclusion}\index{conclusion} of Anne's argument is
the statement, ``Gay marriage should be legal.''  The reason that Anne
gives for this conclusion --- ``Only people with some backwards
religious view believe otherwise'' --- will be called the
\emph{premise}\index{premise} of the argument.  Thus, the argument
consists of a premise, and a conclusion that is supposed to be
supported by the premise.

Thus, we have three things in play: \emph{argument},
\emph{conclusion}, and \emph{premise}.  The argument itself is made up
of the conclusion and the premise.  The conclusion and premise
themselves are particular sentences.  Notice, moreover, that these
sentences are \textit{assertions}, i.e.\ they make a statement that is
either true or false.  Thus, an argument is built out of assertions
(or statements), some of which are premises, and one of which is the
conclusion.

The key point about an argument is that it's more than just a
disconnected collection of statements.  Suppose that I have ten
notecards, each of which has a statement on it.  If I shuffle them up,
and hand them to you, then I haven't given you an \emph{argument}.
For a collection of statements to be an argument, there has to be some
implied sense in which some of the statements stand in a special
relation to another one of the statements.  In the notecard analogy,
I'd have to hand you a first batch of notecards and say, ``these are
my premises,'' and then I'd have to hand you another notecard and say,
``and this statement is my conclusion --- which, I claim, follows
logically from those premises.''

What is this relation of ``following logically'' that I claim holds
between my premises and my conclusion?  We all know it when we see it,
and we have many words for it --- words such as ``supports'' or
``implies'' or ``entails'' or ``shows that'' or ``grounds''.  That is,
we say things like, ``The fact that there are cookie crumbs on the
carpet \textit{shows that} my son was eating in the living room.''

It's this relation --- whatever it is --- that we really want to
understand.  We want to know: when does this relation hold between
statements?  When does one statement imply another?  There is simply
nothing more basic to human rationality than the notion of one
statement implying another.

We will make a lot of progress in clarifying the notion of
implication.  But we're not going to make progress by means of a
head-on assault.  That is, we're not going to offer you a definition
of the form:
\begin{quote} To say that one statement implies another means that
  \dots \end{quote} Such a definition would be interesting, but it's
not what this book is about.  This book is more of a training manual
for logic connoisseurs.  Just as a wine connoisseur knows a good wine
when he tastes one, so a logic connoisseur knows a good argument when
she sees one.


\section{Logical form}

The study of logic began in ancient Greece --- and possibly in other
places at other times, although that history is less well known to us.
It all began with a single insight, which you've probably already had
yourself.  This insight is that whether or not an argument is good
depends only on its \emph{form}, and not on its \emph{content}.  To
explain this distinction, we need to back up for a second and explain
what we mean by saying that an argument is ``good.''  Consider the
following argument:
\[ \begin{array}{p{7cm}}
  All whales are mammals. \\
  David Hasselhoff is a whale. \\
  Therefore, David Hasselhoff is a mammal. \end{array} \] 
Here there are two premises, and one conclusion.  We've used the word,
``therefore'' to indicate what the conclusion is.  But in truth, the
word ``therefore'' isn't part of the content of the conclusion.  The
conclusion is just the proposition, ``David Hasselhoff is a mammal.''

Is this a good argument?  I hope that your answer is, ``it depends.''
It certainly isn't a perfect argument, because it involves a false
statement, namely that David Hasselhoff is a whale.  Or maybe you don't
know anything about David Hasselhoff?  (Such deplorable lack of
cultural knowledge these days!)  Suppose that David Hasselhoff were
actually a famous whale in a book by an obscure author named Melvin
Hermanville.  In that case --- if Hasselhoff were a whale --- then would
it be a good argument?  Yes, it would definitely be a good argument.

If you're a philosophy type, then you might still be doubtful.  You
might be thinking, ``it all depends on what you mean by `good'.''  If
by ``good'' we mean ``interesting, informative, and non-trivial,''
then that argument might not be very good.  However, logic has no use
for subjective words such as ``interesting.''  Logic is the
\textit{science} of good arguments, and it's interested in isolating
an \textit{objective} sense of goodness in arguments.

The insight --- passed on to us by the ancient Greeks --- is that we
can define ``good argument'' in an objective sense by factorizing
goodness into two distinct pieces.  The first of the two pieces is
easy to understand, but difficult to agree upon in practice: are the
premises true?  The second piece is a bit more elusive, but forms the
subject matter of logic as an objective science: do the premises
support (or imply, or entail) the conclusion?  If the premises do
imply the conclusion, then we say that the argument is valid.
\begin{defn}
  An argument is said to be \emph{\gls{valid}} if its premises imply
  its conclusion. \end{defn} The notion of validity isn't concerned
with whether the premises or conclusion are true or false.  The
question, instead, is a conditional one: \textit{if} the premises were
true, \textit{then} would the conclusion be true?

You should be able to think of cases where you would agree that the
premises support the conclusion, even though you think that the
premises are false.  It might help to use the phrase, ``the premises
\textit{would} support the conclusion'', the idea being that if they
\textit{were} true, then they would imply that the conclusion is also
true.

You should also be able to think of arguments where the premises
\textit{and} conclusion are true, but the premises do \textit{not}
imply the conclusion.  For example, the following is a true premise:
``I love coffee.''  The following is also true: ``I am over six feet
tall.''  But to make an argument from my loving coffee to my above
average height would be patently invalid.  Logical validity is all
about the connection between premises and conclusion; it's not
directly concerned with the question of whether the premises or
conclusion are true.

\section{Sameness of form}

How do we get our hands on this elusive notion of validity, and the
related notion of implication?  Let's begin by looking at obvious
cases --- where an argument is obviously valid, or obviously invalid.
For example, the argument above was obviously valid.  But the argument
below is obviously invalid:
\[ \begin{array}{p{7cm}}
  Princeton is a town in New Jersey.  \\ Therefore, God doesn't
  exist. \end{array} \] Now, you might actually think that both of these
statements are true.  But that most certainly doesn't mean that the
first statement implies the second.  Some true statements just don't
have anything to do with each other.  And that's why this argument is
invalid --- because the premise doesn't give the right kind of support
for the conclusion.

Consider another argument:
\[ \begin{array}{p{7cm}}
  All whales are predators. \\
  Bambi is a whale. \\
  Therefore, Bambi is a predator.
   \end{array} \]
 Is that a valid argument?  Before you answer, remember that validity
 doesn't have anything to do with whether you believe the premises or
 the conclusion.  It's merely a matter of whether there is the right
 kind of connection between premise and conclusion.

  Imagine for a moment that you just learned English, and that you
  aren't yet familiar with the word ``whale,'' or with the name
  ``Bambi.''  For all you know, ``whale'' might mean the same thing as
  ``tiger.''  And for all you know, ``Bambi'' might be the name of a
  tiger at the Philadelphia zoo.

  Here's the amazing thing: you don't have to know anything about the
 meaning of the words ``whale,'' ``predator'' or ``Bambi'' to know
 that this argument is valid.  How do you know it's valid?  I'm not
 going to try to answer that question directly.  I'm going to assume
 that you share my intuition that it is obviously valid.  If you're
 still not convinced, let me put it this way:
 \begin{quote} If all whales were predators, and if Bambi were a
   whale, then would it follow that Bambi is a predator?  \end{quote}
 Now it seems pretty obvious, doesn't it?

 We said that the validity of that argument doesn't depend at all on
 what the ``content words'' mean.  In other words, if an argument is
 valid, then it should remain valid no matter how we interpret the
 content words, or even if we replace the content words with different
 ones.  Thus, given a valid argument (such as the one above), we
 should be able to create a sort of ``mad lib argument'' with
 variables that can be filled in by content words.
 \[ \begin{array}{p{6cm}}
      All $X$ are $Y$.  \\ $m$ is an $X$. \\
      Therefore, $m$ is a $Y$. \end{array} \] 
 No matter what words you put in for $X$, $Y$, and $m$ (provided that
 the result is a well formed sentence), you get a valid argument.

 The thing above with the variables, it's like a blueprint for
 constructing arguments.  Choose some content words, plug them in, and
 ta da, you have a valid argument.  Let's call it an \emph{argument
   form}.  In this case, it's a valid argument form, because no matter
 what words you plug in, the argument comes out as valid.

 But how did we know that those arguments were valid in the first
 place?  To be honest, it's just our intuition that tells us that
 these arguments are valid.  Nobody found a tablet of stone on a
 mountain with the argument form above.  Instead, that argument form
 was written down by a human being in an effort to capture what is
 common in a bunch of arguments that we feel (intuitively) to be
 valid.

 That's how we'll proceed in the first part of this book: we will
 collect several basic argument forms that seem obviously valid.  Then
 we'll learn how to string valid argument forms together to create
 longer valid arguments.

%%% Local Variables:
%%% mode: latex
%%% TeX-master: "main"
%%% End:
