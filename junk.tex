In the twentieth century, philosophy students were standardly required
to take a course in symbolic logic.  Along with the standard methods
of proof, these budding philosophers learned a bit about the method of
modeling, and about how it could be used to prove that things cannot
be proven.  Of course, the method of modelling can also be used to
prove that some sentences or beliefs are consistent: if there is a
model $M$ in which all of the sentences are true, then they {\it could
  be} true simultaneously, i.e.\ they are consistent.

This sort of thinking inspired one of the most famous moves in
twentieth century philosophy of religion.  In the mid-twentieth
century, the British philosopher J.L. Mackie argued that it is
logically impossible that God exists.  In particular, Mackie claimed
that the following sentences are logically inconsistent:
\[ \begin{array}{p{7cm}} 1. God is omnipotent. \\ 2. God is
    omnibenevolent. \\ 3. There is terrible suffering. \end{array} \]
Everyone knows that 3 is true, and traditional theists believe that 1
and 2 are true.  If Mackie is correct about the inconsistency of 1,2,
and 3, then traditional theism is false.

Along then came Alvin Plantinga, a philosopher of religion with a nose
for symbolic logic.  Plantinga claimed contra Mackie: there is a model
$M$ in which 1,2, and 3 are true --- and therefore these sentences are
consistent.  Plantinga called his model $M$ the ``free will defense'',
since $M$ describes a world in which God had to create beings with
free will in order to realize the greatest good.\footnote{See Alvin
  Plantinga, \textit{God, Freedom, and Evil}.  Grand Rapids, MI:
  Eerdmans.}

Of course, Plantinga's model $M$ is only a model by analogy.  It's not
a model in the strict sense that logicians use the word, and that we
will be using in this book.  The point, however, is just that the idea
of modelling is a powerful tool in science and in philosophy.
