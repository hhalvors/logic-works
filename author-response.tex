\documentclass[12pt]{article}
\begin{document}

\noindent Response to referee reports.  Hans Halvorson, \emph{How
  Logic Works} \newline 
\noindent \today 

\bigskip \noindent Thank you to the three reviewers for such a careful
reading of the manuscript, and for their sage advice about how to
revise it.  Of course, I am delighted that the reviewers' evaluations
were largely positive; and I believe that I will be able to
incorporate their suggestions to make the book even better.

I will proceed now to list my specific responses to the reviewers'
suggestions.

\subsection*{Reviewer 1}

I gratefully accept the first reviewer's suggestion that I
  change the phrase (on p 30) ``If $P$ is supposed then $P$ is true''
  to ``If $P$ is supposed then $P$ is true according to that
  supposition.''  I agree that the latter is more accurate.

  \subsection*{Reviewer 2}

  The second reviewer gave many helpful suggestions for modifying the
  text.  I agree with all of them.  Many of them are straightforward,
  and I will easily be able to edit the text to accommodate them.  A
  couple of the suggestions call for more thoroughgoing revision, and
  I will now explain how I plan to carry out those revisions.
  
  First, the reviewer points out (correctly, I think) that the
  ``Models'' chapter involves more technical complication than is
  warranted.  In fact, I tested the manuscript on my class at
  Princeton University in spring 2019, and they had precisely the same
  response to this chapter.  As a result, I propose to prune the
  ``Models'' chapter of the more esoteric technical bits.  Thankfully,
  the reviewer has pointed out exactly the sort of thing I can do to
  make the chapter more accessible.  e.g.\ instead of using projection
  maps, I will just use ordered pairs.

  Second, the reviewer pointed out that the final chapter feels a bit
  more disjointed than the earlier ones.  I think that is a fair
  assessment, as it was written in greater haste, and as the material
  had never been tested on my undergraduate class.  Accordingly, I
  propose to revise the final chapter with an eye towards making the
  exposition more straightforward and accessible.

  \subsection*{Reviewer 3}

  The third reviewer also offered several helpful suggestions.  I
  agree with all of them, and many of them will be straightforward to
  implement.  I will only mention two of the suggestions that entail a
  more large-scale revision of the manuscript.

  First, the third reviewer suggests that I add some material to the
  final chapter --- in particular, material about extensions and
  alternatives to classical logic.  I agree with this proposal, but
  that will call for adding approximately three more pages of text.
 
  Second, the third reviewer was somewhat puzzled by the fact that I
  defined validity in terms of the inference rules.  She (or he)
  suggests instead that I define validity in terms of
  truth-preservation.  Interestingly, the second reviewer also had a
  concern about how validity was defined; but he suggests that I
  simply point out more clearly that the inference rules \emph{are}
  the definition of validity.

  I will make some revisions here in order to clarify the definition
  of validity.  However, I do not plan to define validity in terms of
  truth-preservation, because it is precisely one of the novel
  features of this text that it does \emph{not} use that definition.
  I myself have philosophical qualms about that definition, because I
  believe that the notion of truth-preservation is more heavily
  philosophically loaded than the notion of a permitted inference.
  What's more, I want to make students feel the weight of the
  \emph{decision} about which rules to adopt.  I don't want them to
  think that there's some straightforward process of extracting the
  rules of logic from some pre-existing sharp notion of what it means
  to be truth-preserving.  That perspective would be historically
  inaccurate, and might reduce students' creative involvement with the
  process of constructing and evaluating potential rules.

  I realize that my approach to the definition of validity is a bit
  heterodox compared to other textbooks on the market.  Since my book
  is supposed to be an introduction, and not a philosophical
  monograph, I think I should be a bit more gentle in nudging in the
  direction of the view I favor.  To that end, I think it might be
  best to add a few more sentences to the ``for the instructor''
  section to explain my methodology.
  



\end{document}


%%% Local Variables:
%%% mode: latex
%%% TeX-master: t
%%% End:
