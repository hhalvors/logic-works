\documentclass[12pt]{article}
\begin{document}

\thispagestyle{empty}

\noindent \emph{How Logic Works} joins a rather large pantheon of
elementary logic textbooks, but it is different by design. There are
two extremes when it comes to instruction in traditional logic. The
first basically presents the subject with the sparse concision of
mathematics, and is probably best only taught to serious students of
that subject. The second, which comprises most logic textbooks, is
prolix to a fault, and tends to gross oversimplification.  \emph{How
  Logic Works} is concise, but does not skip over the important points
that students of the humanities will find interesting and helpful. Our
working title was, in fact, \emph{Logic For Humans}.  Additionally, it
is designed to transition smoothly into more advanced topics in logic
by teaching general techniques that apply in more complicated
scenarios, such as how to use logic to formulate theories about
specific subject matter.  We also emphasize the limits of logic: the
book is literally about ``how logic works'' as a tool, but a tool
cannot replace its user.  In this book, logic is presented not as new
facts to be learned, but more as a method for organizing the facts
that one already knows, or for new information that will be acquired
as the march of life goes on.

\end{document}


%%% Local Variables:
%%% mode: latex
%%% TeX-master: t
%%% End:
