\chapter{Beyond Logic}

Since logic has no content, you cannot have failed to learn the
content of this book.  We hope, though, to have shown you an example
of how you can think more clearly, more rigorously, and more freely.

You might be disappointed.  You might have hoped that logic would tell
you what to to believe, or how to behave.  However, since logic only
cares about form (and not content), it cannot possibly advise us on
what to believe.  At best, logic can help us to calculate the costs of
our beliefs.  And once again, it's up to you to decide what costs you
are willing to pay.

Consider, for example, the following simple argument for God's
existence.
\[ \begin{array}{p{10cm}}
     If God does not exist, then there are no moral rules. \\
     There are moral rules. \\
     Therefore, God exists. \end{array} \] This argument is valid.
  But so what?  It doesn't tell you what to believe.  Perhaps you
  don't believe the premises.  Or perhaps you believe the premises;
  and upon discovering that they entail this conclusion, you'll decide
  to reject the premises.  Logic does not tell you that you ought not
  do that.  There's an old philosophers' saying: {\it one person's
    modus ponens is another person's modus tollens}.  In other words,
  logic doesn't tell you whether to accept the premises and the
  conclusion, or whether to reject one of the premises because you
  reject the conclusion.

\section{What next?}

Did you just waste a class (or many hours, or both) on learning
techniques that you'll never again use?  Even if you never again write
a formal proof, the work you've put into learning formal logic will
not be wasted.  Consider an analogy.  A competitive athlete may spend
hour upon hour performing exercises that she will never perform in
competition.  Of course, those exercises are not wasted time.  The
different individual exercises are like vectors that can be summed
together to produce the desired outcome during competition.  The
individual component contributors to performance may be invisible; but
if they weren't there, the performance would be undermined.

Your life is more important than any athletic competition, and your
brain is one of your most important tools for winning in life (however
you define that for yourself).  You can think of formal logic as the
brain's version of isolation exercises.  By learning the individual
inference rules, and by using them again and again, you've built some
exquisite mental muscles.  In real life, you might never have occasion
to employ these individual mental muscles in isolation.  However,
anytime you need to think hard, or fast, or clearly, these individual
mental muscles will combine to enable you to perform at the highest
possible level.

If you want to go further with formal logic, then I have good news for
you: it's a thriving subject, with connections to many other fields of
study, such as computer science.  As for the study of logic itself,
there are many different directions you could go from here, and I'll
briefly discuss five of them.

First, you might wish to study \emph{extensions of classical logic}.
For philosophers, the most important of these extensions is
\emph{modal logic}, which studies intensional connectives such as ``it
is necessarily true that \dots '' usually symbolized with a box
$\Box$.\footnote{For propositional modal logic, see J.C. Beall and
  B. van Fraassen, {\it Possibilities and Paradox}, Oxford (2003), or
  G. Forbes, {\it Modern Logic}, Oxford (1994).  For quantified modal
  logic, see K. Konyndyk, {\it Introductory Modal Logic}, University
  of Notre Dame Press (1986), or T. Sider, {\it Logic for Philosophy},
  Oxford (2010).}

While modal logic has primarily found its audience among philosophers,
other extensions of classical logic are of interest in the exact
sciences.  Here we should mention higher-order logics (where one can
quantify over
subsets)\footnote{https://plato.stanford.edu/entries/logic-higher-order},
infinitary logics (where one can form infinite conjunctions or
disjunctions of sentences), and the lambda calculus (where one adds an
operator for forming names out of predicate
phrases)\footnote{https://plato.stanford.edu/entries/lambda-calculus}.

Second, if you're feeling a bit more revolutionary, then you might be
interested in studying \emph{alternatives to classical logic}.  These
alternatives to classical logic can again be subdivided into two
classes: fragments of classical logic, and substructural logics.  A
fragment of classical logic is a logic that uses only some subset of
the logical vocabulary or the inference rules.  For example,
\emph{intuitionistic logic} drops the double-negation elimination
rule, and instead adopts an ex falso quodlibet rule.  (In this case,
excluded middle can no longer be proven.)  The move to intuitionistic
logic was initially motivated by an outlook in the philosophy of
mathematics that has largely been discredited.  However,
intuitionistic logic is still an important tool for reasoning about
mathematical structures that do not ``live in'' the universe of
sets.\footnote{S. Mac Lane and I. Moerdijk, {\it Sheaves in Geometry
    and Logic}, Springer (1994).}  More generally, \emph{coherent
  logic} drops the negation symbol and the universal quantifier, and
so is neutral between intuitionistic and classical logic.

A \emph{substructural logic} is a logic that modifies some of the
rules we tacitly adopted for manipulating dependency numbers.  In
particular, we tacitly assumed that lists of dependency numbers
aggregate like sets, e.g.\ the aggregate of ``$2$'' and ``$2,3$'' is
``$2,3$'', which is no different than ``$3,2$''.  In substructural
logic, these identities are no longer assumed to hold.  Already in the
1960s, some logicians argued that changes in the structural rules were
the best solution to the paradoxes of material
implication.\footnote{https://plato.stanford.edu/entries/logic-relevance}
More recently, it has been observed that changes in the structural
rules can yield logics that better represent the kind of reasoning
used in quantum physics (e.g.\ quantum logic)\footnote{P. Gibbins,
  {\it Particles and Paradoxes: The Limits of Quantum Logic},
  Cambridge University Press (1987).}, and in computer science (e.g.\
linear logic).\footnote{A.S. Troelstra, {\it Lectures on Linear
    Logic}, CSLI (1992). For an general overview, see G. Restall, {\it
    An Introduction to Substructural Logics}, Routledge
  (2000).} \label{relevant}

Third, this entire book has focused on a limiting case of good
arguments, namely those arguments where the premises provide decisive
support for the conclusion (i.e.\ deductively valid arguments).  So,
it would make a lot of sense to now go on to study less idealized
cases, where the premises are only intended to provide some (less than
decisive) support conclusion.  One promising approach to this idea is
to use the \emph{probability calculus}, which offers various ways to
measure the evidential support that premises provide for a
conclusion.\footnote{Colin Howson and Peter Urbach, {\it Scientific
    Reasoning: The Bayesian Approach}, Open Court (2005).}  More
generally, various \emph{inductive logics} have been proposed,
although there has been some controversy among philosophers about
whether the notion of inductive support can be properly
formalized.\footnote{Brian Skyrms, {\it Choice and Chance: An
    Introduction to Inductive Logic}, Cengage (1999).}

Fourth, one could proceed from here to a more in depth study of the
metatheory of first-order logic.  For example, in \emph{proof theory}
one builds and studies elegant ``sequent calculi.''  In fact, we
intentionally chose the proof system in this book because it closely
resembles the sequent calculus, and so anyone who learns this system
is well-prepared to move on to proof theory.\footnote{A.S. Troelstra
  and H. Schwichtenberg, {\it Basic Proof Theory}, Cambridge (2000).}
Going in a different meta-theoretical direction, in \emph{model
  theory} one studies the relation between theories and their models,
and its here that one proves some of the most powerful results of
metalogic.\footnote{D. Marker, {\it Model Theory: An Introduction},
  Springer (2002).} For example, the L{\"o}wenheim-Skolem theorem
shows that any theory with an infinite model also has a countably
infinite model --- which is deeply puzzling when applied to
Zermelo-Fraenkel set theory, which entails that there is an
uncountably infinite set.\footnote{For a general overview of
  metatheory, see G. Hunter, {\it Metalogic: An Introduction to the
    Metatheory of Standard First Order Logic}, University of
  California Press (1996), or H. Enderton, {\it A Mathematical
    Introduction to Logic}, Academic Press (2001).}

Fifth, you might wish to study particular theories within first-order
logic.  Of course, that's precisely what's done in many different
parts of mathematics --- e.g.\ one studies group theory, or ring
theory, or field theory, or \dots . However, some such theories are of
special interest to logicians, most particularly Zermelo-Fraenkel set
theory and Peano arithmetic.  Set theory has itself become a massive
field of study, and there are many good textbooks.  As for the study
of Peano arithmetic and G{\"o}del's incompleteness theorem, we would
point the interested reader to Burgess, Boolos, and Jeffrey, {\it
  Computability and Logic}, Cambridge (2007), or P. Smith, {\it An
  Introduction to G\"odel's Theorems}, Cambridge (2013).

Sixth, and finally, you might want to study the network of all
theories as they are related to each other via translations.  Here we
would (immodestly) point you to H. Halvorson, {\it The Logic in
  Philosophy of Science}, Cambridge (2019).

%% it cannot tell you what to believe

%% people think that we need to move to intensional logics.  But even
%% intensional logics are formal logics ... only capture formal relations

%% Carnap: principle of tolerance? 






%%% Local Variables:
%%% mode: latex
%%% TeX-master: "main"
%%% End:
